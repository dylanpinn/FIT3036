\documentclass[a4paper,11pt]{article}

\usepackage[utf8]{inputenc}
\usepackage[british]{babel}
\usepackage{csquotes}
\usepackage[style=apa]{biblatex}
\usepackage{graphicx}
\usepackage{float}
\usepackage{algorithm}
\usepackage[noend]{algpseudocode}

\bibliography{report}

\title{FIT3036 --- Final Report}
\author{Dylan Pinn --- 24160547 \thanks{dcpin3@student.monash.edu}}
\date{2018, May}
% Provide the title, abstract, author names (and contact details,
% affiliations, etc.) and keywords on the first page alone (i.e. this is a
% separate cover page).
% Abstract (<300 words)
% Keywords (<6)

\begin{document}
\maketitle
\vspace{-1ex}
\begin{center}
  % TODO: Update this at the end.
  Word Count: 1416
\end{center}

\begin{abstract}
\end{abstract}
\pagebreak

\tableofcontents
\pagebreak

\section{Introduction}
% TODO: Add starting Introduction

\subsection{Project Objectives}

Out company secured a contract for a local council in Victoria to resurface
roads in a designated square kilometre area. The objectives are to design,
implement, test and deliver a system that will perform these calculations and
display the result to the user.

\subsection{Requirements}

We are to use Google Maps and related products and any satellite/aerial views.
The project involves writing the related code, with an elegant GUI to support
calculations. \autocite[2]{intro:1}

\subsection{Constraints}

We have the following constrains on the project:

\begin{itemize}
  \item Only have access to publicly available data.
  \item Limited to available information online.
  \item No/Limited access to council records.
  \item The project must be finalised by the end of week 12.
\end{itemize}

\section{Background}
% TODO: Add section intro.

\subsection{Project Info}
% TODO: Add subsection intro.

\subsubsection{Project Risks}

We have identified the main project risks and their probability.

\begin{itemize}
  \item Cannot find appropriate road data for calculations. (Medium)
  \item Cannot access data from the OpenStreetMap API.\@ (Low)
  \item Google Maps API offline. (Low)
  \item Problems with infrastructure. (Low)
  \item Completing the project on-time. (Medium)
\end{itemize}

\subsubsection{Risk Reduction Strategies}

Using the following risk reduction strategies we have been able to reduce these
risks and lower their probability.

\begin{itemize}
  \item Research sources for street data from various sources.
    \begin{itemize}
      \item Google Maps did not provide enough data to be useful.
      \item OpenStreetMap provided the required data and an API to access it.
    \end{itemize}
  \item Planning out the project; breaking it up into small components that can
    be worked on sequentially in a short period of time.
\end{itemize}

\subsubsection{Risks Encountered}

The following issues arose throughout the project.

\begin{itemize}
  \item Finding data sources
  \item Finding time to complete the project while working full time and
    completing assessments for other subjects.
\end{itemize}

\subsection{Resource Requirements}
% TODO: Add subsection intro.

\subsubsection{Hardware}

We have decided to use as much Software as a Service (SaaS), Platform as a
Service (PaaS) and Infrastructure as a Service (IaaS) solutions to keep costs
down. \autocite[4]{project-spec:2}

\paragraph{Backend}

To host the backend API, Amazon Web Services (AWS) has been chosen due to its
comprehensive free tier and prior experience with the platform. Integration with
the platform is handled by the Serverless framework \autocite{serverless:3}. The
following AWS services have been used:

\begin{itemize}
  \item AWS Lambda
    \begin{itemize}
      \item Functions as a Service (FaaS).
      \item Allows running backend code without maintaining server
        infrastructure \autocite{aws:4}.
    \end{itemize}
  \item AWS API Gateway --- Allows creating public API endpoints for Lambda
    functions \autocite{aws:5}
\end{itemize}

\paragraph{Frontend}

To host the frontend application, we have chosen Netlify \autocite{netlify:6},
it provides free services to open source projects. This includes a CDN, static
file hosting and automatic deployments from CI/CD providers.

\paragraph{Other Services}

To follow devops best practices we are automating the running of our tests
and deployments. This is managed through the Travis CI service. They were
chosen because they also offer free services for open source projects
\autocite{trci:7}.

\subsubsection{Software}

The following software was chosen to aid in the development of the project.

\paragraph{Backend}

The backend API was developed using the Go programming language
\autocite{go:8}, this was chosen because of its speed, as it is a compiled
language \autocite{plural:9}. This will allow it to parse and manipulate large
amounts of data quickly.

\paragraph{Frontend}

The frontend application was created using React, a frontend JavaScript
framework created by Facebook \autocite{react:10}. This was chosen as it
allows us to easily create user interfaces that are testable.

\paragraph{Documentation}

The documentation is written in \LaTeX, this was chosen for several reasons.

\begin{itemize}
  \item We don't have to worry about the typesetting of the document. \LaTeX{}
    takes care of all of the formatting.
  \item Includes a bibliography package BibTex.
  \item Allows all of the documentation to be version controlled.
\end{itemize}

\subsection{Tasks}

The following tasks were planned and executed to complete the project:

\begin{itemize}
  \item Project Specification
  \item Project
    \begin{itemize}
      \item Backend API
      \item Frontend Application
    \end{itemize}
  \item Presentation
  \item Test Plan
  \item Workbook
  \item Final Report
\end{itemize}

% TODO: Add Gantt charts and other project management information.

\section{Method}

% TODO: Add intro to section.

\subsection{Methods Overview}
% TODO: Add intro to subsection.

\subsubsection{Calculate Area}
% TODO: Add intro to subsection.

\begin{itemize}
  \item The user uses the map controls in the frontend application to refine
    the desired area to calculate the surface area of the roads.
  \item The coordinates of the square is passed via a HTTP POST request to
    the backend API.\@
  \item These coordinates are used to calculate the area of the square.
  \item This is done by calculating the distance between between the
    north/south coordinates and the east/west coordinates.
  \item This is then multiplied together to return the area.
  \item This result is displayed for the user in the user interface.
\end{itemize}

\subsubsection{Calculate Surface Area of Roads}
% TODO: Add intro to subsection.

\begin{itemize}
  \item Once the user is satisfied with the area and clicks the ``Calculate
    Area'' button it performs a HTTP POST request to the backend API.\@
  \item The coordinates of the rectangle are sent in the POST body as JSON.\@
  \item These coordinates are used by the backend API to generate a query to
    send to OpenStreetMap via their OverPass API.\@
  \item This returns all of the road information contained within the rectangle
    bounds.
  \item The roads are then looped over, calculating the distance between each of
    the nodes of the road. Once the distance is calculated it uses the lane
    information if provided otherwise assumes that the road contains 2 lanes.
  \item The distance is multiplied by the number of lanes and the width of a
    lane, 3.5m the default lane width within Australia \autocite{lane:11}.
\end{itemize}

\subsection{Internal Design}

% TODO: Add intro to subsection.
% TODO: Add Sequence diagrams.

\paragraph{Calculate Area of a Rectangle}

% TODO: Connect to Figure.
These steps below are outlined in FIGURE

\begin{itemize}
  \item User moves the map, adjusts zoom level or changes input.
  \item The bounds of the rectangle are send to the backend service.
  \item This calculates the total area of the bounds.
  \item This is returned to the user.
\end{itemize}

\paragraph{Calculate Surface Area of Roads}

% TODO: Add Sequence diagrams.
% TODO: Connect to Figure.
These steps are outlined in FIGURE

\begin{itemize}
  \item User requests to calculate total surface area of roads.
  \item These parameters are sent to the backend service.
  \item These parameters are used to call OpenStreetMap API.\@
  \item It then iterates over the returned data and calculates the distance of
    all of the roads.
  \item If any of the points of the data are outside the bounds of the rectangle
    they are ignored.
  \item The area of the roads is then calculated with the lane width.
  \item These results are summed.
  \item This is returned to the user.
\end{itemize}

\subsection{Software Architecture}

FIGURE shows the architecture of the backend application. It is broken up into 4
main classes, which calculate area, sum data, calculate distance and calculate
the area of the roads.

% TODO: Add class diagram
% TODO: Update code or diagrams to match description.

\subsection{Algorithms}

% TODO: Add intro
% TODO: Add pseudocode for key methods.

\section{Results}
% Results
% This is the major new contribution in the report. In general, it is for you to analyse the outcomes (and quality) of the project.
% What sort of results or output were obtained with your program?
% If your project itself contains original research, elaborate on that as well.
% If your project deals with experimental statistics/case studies, kindly note the following points:
% describe the experiments performed;
% the statistics/outputs collected;
% Externally observable features of your software/simulation: input, output.
% Performance: include time and space performance characteristics of your software as well as estimated complexities (including real-world processing time, storage space, bandwidth, etc.)

% TODO: Add intro.

The Program has 2 main outputs:

\begin{itemize}
  \item Area of the bounds.
  \item Total surface area of the road between the bounds.
\end{itemize}

The bounds is the rectangle area comprising of a set of coordinates; North,
South, East and West.

Finding reliable data to test against is hard. Writing unit tests allow testing
smaller parts of the application in isolation. This helps give confidence to the
accuracy of the application.

OpenStreetMap API returns data from outside the requested bounds. As the data
points can start within the bounds and finish outside it is excluded. This
reduces the accuracy.

Lane information isn't always included in the OpenStreetMap API so assumptions
are made about the default number of lanes. We have chose a safe default that if
no lane information is provided then the road is 2 lanes wide.

Using a default lane width of 3.5m, also reduces the accuracy. If we can find a
dataset that contains the lane width of each road with the vector data this
would help improve the accuracy of the program.

In the initial research, we couldn't find any other implementations/research
that did these calculations using vector data. Because of this the methodology
of the application was implemented through trial and error.

We created throwaway spikes to test implementation of the backend API.\@ This
included querying a small set of data and performing the calculations on it.
This helped to fine-tune and improve the accuracy of the algorithms used.

\subsection{Externally Observable Features}

% TODO: Add intro

\subsubsection{Input}

\begin{itemize}
  \item User can manipulate the Google Map component using either touch or
    mouse.
  \item User can input coordinates for the Map to centre on this position by
    using the latitude and longitude number inputs.
\end{itemize}

\subsubsection{Output}

\begin{itemize}
  \item The area in km\textsuperscript{2} of the requested area is displayed to
    the user.
  \item This is a hard-coded square of size 1km\textsuperscript{2}.
  \item The user then can request the total surface area of the roads within
    this area.
  \item This result is presented to the user in the frontend interface.
\end{itemize}

\subsection{Performance}

The backend API is hosted on AWS Lambda, this restricts the amount of time the
functions and run for and the amount of space that they can occupy.

The current restrictions are controlled in AWS, the settings are:

\begin{itemize}
  \item 1024 MB of memory.
  \item 6 seconds to run function before timing out.
\end{itemize}

\subsubsection{Area Function}

\begin{itemize}
  \item Average Invocation Time --- 3ms
  \item Average Memory Usage --- 30MB
\end{itemize}

\subsubsection{Road Surface Function}

\begin{itemize}
  \item Average Invocation Time --- 1800ms
  \item Average Memory Usage --- 28MB
\end{itemize}

\section{Analysis \& Discussion}
% Analysis & Discussion
% analyse those statistics/outputs.
% e.g. statistics-based project: results/outcomes for statistical tests on your data; evaluate the quality of your output; what statistics were collected under what circumstances.
% e.g. qualitative project/based on case studies: what case study is chosen; what is the motivation behind such a study; your experimental outcomes; critiques on the outcome; comment on existing studies; also literature/sources that back up your claims.
% Report how the statistics/outputs support or undermine the main hypotheses behind this project. What hypotheses were supported/disconfirmed by the data?
% If you don’t have any idea how this kind of research might be reported, look at a few research articles in the journal JASSS: http://jasss.soc.surrey.ac.uk/JASSS.html

\section{Future Work}
% Future work: Include also any possible future research that might emanate from the project.

\section{Conclusion}

\section{Bibliography}

\printbibliography{}

\section{Appendices}
% Appendices. Including, as advised by your supervisor:
% Production and Deployment [IMPORTANT]: detailed instructions to run/install/compile your software, with complete instructions/screenshots. Include ALL information such as dependencies, platform, etc.
% Parameters used, their ranges. If GUIs were used: GUI features used for them (e.g. input boxes, choices, etc...) If a command interface was used: parameters, switches, arguments, etc.
% User Interface: How to run the software (be as specific as possible). Any other functions made available to the user (incl. screenshots of the various features "at work" with sample data). Also, for WWW-based services/appliances - detail clearly the interface(s) to your Web-based project.
% Any externally available functions not covered above (including debugging tools, auxilliary/helper functions, bonus features...
% Internal testing procedures implemented including cross-references to your Test Report (which is submitted separately)
% etc.

\end{document}
