\documentclass[a4paper,11pt]{article}

\usepackage[utf8]{inputenc}
\usepackage[british]{babel}
\usepackage{csquotes}
\usepackage[style=apa]{biblatex}
\usepackage{graphicx}
\usepackage{float}
\usepackage{algorithm}
\usepackage[noend]{algpseudocode}

\bibliography{report}

\begin{document}

\title{FIT3036 --- Final Report}
\author{Dylan Pinn --- 24160547}
% Provide the title, abstract, author names (and contact details,
% affiliations, etc.) and keywords on the first page alone (i.e. this is a
% separate cover page).
% Show Word Count on title page.
% Abstract (<300 words)
% Keywords (<6)
\maketitle
\pagebreak

\tableofcontents
\pagebreak

\section{Introduction}
% TODO: Add starting Introduction

\subsection{Project Objectives}

Out company secured a contract for a local council in Victoria to resurface
roads in a designated square kilometre area. The objectives are to design,
implement, test and deliver a system that will perform these calculations and
display the result to the user.

\subsection{Requirements}

We are to use Google Maps and related products and any satellite/aerial views.
The project involves writing the related code, with an elegant GUI to support
calculations. \autocite[2]{intro:1}

\subsection{Constraints}

We have the following constrains on the project:

\begin{itemize}
  \item Only have access to publicly available data.
  \item Limited to available information online.
  \item No/Limited access to council records.
  \item The project must be finalised by the end of week 12.
\end{itemize}

\section{Background}
% Background
% include a review of academic literature used when preparing for the project.
% any research leading up to enabling the current project to happen (e.g.
% current literature, existing case studies, existing studies)
% datasets used in your current project, including any academic literature on the background of the dataset (e.g. a paper by the dataset's authors or info on how it was made)
% project info: main project risks, their probabilities and risk reduction strategies, risks actually encountered
% resource requirements (hardware and software)
% tasks required and their dependency relations in producing the final project; as well as a timeline of the final completed project.

\section{Method}
% Method
% explain the methodology used in your project - provide technical details.
% internal design: description of how the parts of the final completed project are meant to work together (e.g. sequence diagrams, use-case, package overview, etc.)
% software architecture: overall structure of the code; high-level structure (if any) of the final completed project. Consider including class diagrams if you’re using an OO programming language.
% also, elaborate on key algorithms/functions/formulae used – pseudocode, flowcharts, etc.
% statistics (*if your project deals with raw statistics incl. simulations.) -- What statistics are collected and how they are output.

\section{Results}
% Results
% This is the major new contribution in the report. In general, it is for you to analyse the outcomes (and quality) of the project.
% What sort of results or output were obtained with your program?
% If your project itself contains original research, elaborate on that as well.
% If your project deals with experimental statistics/case studies, kindly note the following points:
% describe the experiments performed;
% the statistics/outputs collected;
% Externally observable features of your software/simulation: input, output.
% Performance: include time and space performance characteristics of your software as well as estimated complexities (including real-world processing time, storage space, bandwidth, etc.)

\section{Analysis \& Discussion}
% Analysis & Discussion
% analyse those statistics/outputs.
% e.g. statistics-based project: results/outcomes for statistical tests on your data; evaluate the quality of your output; what statistics were collected under what circumstances.
% e.g. qualitative project/based on case studies: what case study is chosen; what is the motivation behind such a study; your experimental outcomes; critiques on the outcome; comment on existing studies; also literature/sources that back up your claims.
% Report how the statistics/outputs support or undermine the main hypotheses behind this project. What hypotheses were supported/disconfirmed by the data?
% If you don’t have any idea how this kind of research might be reported, look at a few research articles in the journal JASSS: http://jasss.soc.surrey.ac.uk/JASSS.html

\section{Future Work}
% Future work: Include also any possible future research that might emanate from the project.

\section{Conclusion}

\section{Bibliography}

\printbibliography{}

\section{Appendices}
% Appendices. Including, as advised by your supervisor:
% Production and Deployment [IMPORTANT]: detailed instructions to run/install/compile your software, with complete instructions/screenshots. Include ALL information such as dependencies, platform, etc.
% Parameters used, their ranges. If GUIs were used: GUI features used for them (e.g. input boxes, choices, etc...) If a command interface was used: parameters, switches, arguments, etc.
% User Interface: How to run the software (be as specific as possible). Any other functions made available to the user (incl. screenshots of the various features "at work" with sample data). Also, for WWW-based services/appliances - detail clearly the interface(s) to your Web-based project.
% Any externally available functions not covered above (including debugging tools, auxilliary/helper functions, bonus features...
% Internal testing procedures implemented including cross-references to your Test Report (which is submitted separately)
% etc.

\end{document}
